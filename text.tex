%!TEX root = Kurs1884.tex
%!TEX encoding = IsoLatin
%%%%%%%%%%%%%%%%%%%%%%%%%%%%%%%%%%%%%%%%%%%%%%%%%%%%%%%%%%%%%%%%%%%%%%
% Kurseinheit 1
%%%%%%%%%%%%%%%%%%%%%%%%%%%%%%%%%%%%%%%%%%%%%%%%%%%%%%%%%%%%%%%%%%%%%%

\vspace{-.5cm}
\chapter{Grundlagen und Entwurfstechniken}\label{grundlagen} 
\vspace{-.5cm}
In dieser Kurseinheit widmen wir uns dem Interaktionsdesign und lernen zentrale Begrifflichkeiten der Nutzer kennen.
Designer sind auch nur Menschen.
Wie bitte? Nein. 
Begriffe und Klassifikationsschemata von kooperativen Systemen müssen Sie kennen. Sie lernen das Konzept des Entwurfsmusters kennen und entwickeln Strategien zur partizipativen Systemgestaltung mit Entwurfsmustern und papierbasierten Prototypen. Abschließend werden wir in einem fünften Teil auf die ethische Verantwortung bei der Gestaltung eines kooperativen Systems eingehen.\blfootnote{Stand: \today.}

\minitoc	
\vfill
\section*{Lernziele}
Nach der Bearbeitung dieser Kurseinheit sollten Sie in der Lage
sein,
\begin{itemize}
\item[\goal] Beispiele für und Klassifikation von kooperativen Systemen zu benennen,
\item[\goal] die Bedeutung von Raum und Zeit für kooperative Systeme zu beschreiben,
\item[\goal] soziotechnische Systeme von rein technischen Systemen zu unterscheiden,
\item[\goal] theoretische Hintergründe für das Konzept des Entwurfsmusters sowie die Elemente eines Entwurfsmusters zu
benennen,
\item[\goal] Vorteile der partizipativen Gestaltung von kooperativen Systemen zusammenzufassen,
%\item[\goal] den Oregon Software Development Process als ein Beispiel eines musterorientierten Entwicklungsprozesses für soziotechnische
Systeme zu beschreiben, 
\item[\goal] papierbasierte Prototypen zu erstellen und
\item[\goal] ethische Fragestellungen bei der Gestaltung eines kooperativen Systems zu beurteilen.
\end{itemize}

\section{Einleitung}
Inzwischen ist auch in den öffentlichen Debatten die Digitalisierung der Gesellschaft ein allgegenwärtiges Thema. Menschen treten miteinander in sozialen Netzwerken in Kontakt, sie äußern Meinungen und teilen Ideen. Gemeinsam erstellen Tausende von \textquote{Weisen} eine weltumspannende Enzyklopädie mit dem Wissens der Menschheit. In der Industrie arbeiten Ingenieurinnen und Ingenieure in weltweit vernetzten Projektgruppen. Manche Städte erwachsen zu Smart Cities. Alltagsgegenstände vernetzen sich in Smart Homes. Kinder spielen in virtuellen Welten. Nachrichten über Katastrophen verteilen sich über Kurznachrichtendienste in Bruchteilen von Sekunden rund um den Globus. Private Glücksmomente bis hin zum eigenen Herzschlag werden mit den engsten Partnerinnen und Partnern geteilt. 
